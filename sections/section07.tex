
As discussed above, it is not true that tensoring preserves all exact sequences. We want to consider when it is.

\begin{definition}
    An $R$-module $M$ is called flat if for every exact sequence $A \longrightarrow B \longrightarrow C$, the sequence $A \otimes M \longrightarrow B \otimes M \longrightarrow C \otimes M$ is exact.
\end{definition}

For example, the $\mathbb Z$-module $\mathbb Z/(2)$ is not flat, as the sequence
\begin{tikzcd}
    0 \arrow[r] &\mathbb Z \arrow[r, "\cdot 2"] &\mathbb Z
\end{tikzcd}
is not exact when tensored with $\mathbb Z/(2)$.

\begin{remark}
    Any free module in flat. Indeed, 
    \begin{align*}
        (A \longrightarrow B \longrightarrow C) \otimes R^{\oplus I} &= A^{\oplus I} \longrightarrow B^{\oplus I} \longrightarrow C^{\oplus I}\\
        &= (A \longrightarrow B \longrightarrow C)^{\oplus I}
    \end{align*}
    which is exact. More generally, one can show that any direct summand of a flat module is flat, so projective modules are flat. However, the converse is not true. Indeed, $\mathbb Q$ is not $\mathbb Z$-projective but is flat.
\end{remark}

\begin{lemma}
    Let $M$ be an $R$-module. The following are equivalent.
    \begin{enumerate}
        \item $M$ is flat.
        \item Tensoring by $M$ preserves injections.
        \item For any ideal $I$, tensoring by $M$ preserves injectivity of the inclusion of $I$.
    \end{enumerate}
\end{lemma}
\begin{proof}
    1) $\implies$ 2) $\implies$ 3) is immediate. 3) $\implies$ 1) is not, and we will delay this until we have the Tor functor.
\end{proof}