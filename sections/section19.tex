
We begin with a key technical result for dimension counting.
\begin{theorem}[Krull's Principal Ideal Theorem]
    Let $R$ be a noetherian ring, $x \in R$. Then all minimal prime ideals containing $(x)$ have codimension at most 1.
\end{theorem}
\begin{proof}
    Take a minimal prime ideal $\frak p \supseteq (x)$. We want to show that $\dim(R_{\frak p}) \leq 1$. Let $S = R_{\frak p}$. Then $S$ is a noetherian local ring containing $x$. By correspondence, the maximal ideal $\frak m \subseteq S$ is a minimal prime ideal containing $(x)$, so it is the only prime ideal containing $(x)$. This tells us that $\rad((x)) = \frak m$. Let $\frak q \subsetneq \frak m$ a prime ideal. If this doesn't exist, we have $\dim(S) = 0$ so we're done. Else, it suffices to show that $\frak q$ has codimension 0. As $\rad((x)) = \frak m$, we have $\spec(R/(x)) = \spec(R/\frak m)=$ one point. Then $S/(x)$ is local of dimension 0. As it's noetherian, it is also artinian. We must now define the symbolic powers of an ideal.\\
    We define the $n^{th}$ symbolic power of $\frak q$ to be the inverse image of $\frak q^n S_{\frak q}$ along the map $S \longrightarrow S_\frak q$. We denote this $\frak q^{(n)}$. Clearly, $\frak q^n \subseteq \frak q^{(n)}$.\\
    Returning to the problem, we have $\rad(x) = \frak m \supsetneq \frak q$. Then $x \notin \frak q$, so $x$ is a unit in the localization at $\frak q$. Furthermore, we have the descending chain $(x) + \frak q^{(1)} \supseteq (x) + \frak q^{(2)} \supseteq \dots$. By artinianitudenessity of $S/(x)$, this must terminate. Thus, there exists an $n$ such that $(x) + \frak q^{(n)} = (x) + \frak q^{(n+1)}$. Then any $f \in \frak q^{(n)}$ equals some $ax + g$ for $a \in S$ and $g \in \frak q^{(n + 1)}$. Then $ax = f - g \in \frak q^{(n)}$. As $x$ is a unit in the localization, $a \in \frak q^{(n)}$. Then as $x \in \frak m$, we have $\frak q^{(n)} = \frak q^{(n + 1)} + \frak m \frak q^{(n)}$. This tells us that $\frak q^{(n)}/(\frak q^{(n + 1)} + \frak m \frak q^{(n)}) = 0$, so $(\frak q^{(n)}/\frak q^{(n + 1)}) \otimes_S S/\frak m = 0$. By Nakayama's lemma, this tells us that $\frak q^{(n)} = \frak q^{(n + 1)}$. Any ideal in $S_\frak q$ is generated by its intersection with $S$, so this inequality implies that $\frak q^n S_\frak q = \frak q^{n + 1} S_\frak q$. Then $\frak q^n \otimes_{S_\frak q} S_\frak q / \frak q S_\frak q = 0$, so by Nakayama's lemma, $\frak q^n S_\frak q = 0$ This tells us that the unique maximal ideal of the local ring $S_\frak q$ is nilpotent, so we may repeat the proof of artinian = noetherian plus dimension 0 and the fact that $S_\frak q$ is noetherian to conclude that $S_\frak q$ is noetherian, hence 0 dimensional.
\end{proof}

\begin{corollary}
    Let $R$ be noetherian with $x_1, \dots, x_n \in R$. Then any minimal prime ideal containing $(x_1, \dots, x_n)$ has codimension at most $n$.
\end{corollary}
\begin{proof}
    Indeed, let $\frak p \supseteq (x_1, \dots, x_n)$ be a minimal prime ideal. Let $\frak q \subsetneq \frak p$ be prime and maximal with respect to this property. Then some $x_i \notin \frak q$. Without loss of generality, $x_1 \notin \frak q$. Computing the codimension of $\frak p$ comes down to computing the dimension of $R_\frak p$, which is local with maximal ideal $\frak m = \frak p R_\frak p$. Then $\frak p = \rad(x_1, \dots, x_n)$. Furthermore, we have that $\rad(\frak q + (x_1)) = \frak p$. Then some $x_i^{r_i} = q_i + r_i x_1$, $q_i \in \frak q$, $r_i \in R$. In $R/(y_2, \dots, y_n)$, Now take a minimal prime over the image of $(x_1)$. Its pullback to $R$ then contains $x_1, x_i^{r_i}$ for $i \geq 2$. Then, by minimality, it must be $\frak p$, so $\frak p$ is a minimal prime over $(x_1)$ in $R/(y_2, \dots, y_n)$. Then its codimension here is at most 1 by Krull's principal ideal theorem, so the codimension of $\frak q$ in the image is 0. Hence, it is a minimal prime over $(y_2, \dots, y_n)$, so by induction its codimension is at most $n - 1$. Then the codimension of $\frak p$ is at most $n - 1$.
\end{proof}

\begin{corollary}
    Every noetherian local ring has finite dimension.
\end{corollary}
\begin{proof}
    Let $\frak m$ be its maximal ideal. As $R$ is noetherian, $\frak m = (x_1, \dots, x_n)$. By the above corollary, $\codim(\frak m) \leq n$, so $\dim(R) \leq n$.
\end{proof}

\begin{remark}
    This fact was used implicitly in the above proof, but it's helpful to state explicitly that $\dim(R)$ is the supremum of the codimensions of maximal ideals.
\end{remark}

We have then that noetherian rings are locally finite dimensional, but this turns out to not be a local property, even for noetherian rings. Indeed, there are noetherian rings that are not finite dimensional. For example, take $R = k[x_1, x_2, \dots]$. Let $\frak p_1 = (x_1), ]\frak p_2 = (x_2, x_3), \frak p_3 = (x_4. x_5, x_6, x_7), \dots$. Let $S = \bigcup_{n \geq 1} \frak p_n$. Then take $R[S^{-1}]$.

\begin{definition}
    A ring $R$ is called catenary if for any $\frak p \subseteq \frak q$ prime ideals there exists a maximal chain $\frak p_0 \subsetneq \dots \subsetneq \frak p_r = \frak q$ with $r$ unique.
\end{definition}

\begin{theorem}
    Let $R$ be a catenary domain of finite dimension. Then for any prime ideal $\frak p$, $\dim(R) = \codim(\frak p) + \dim(R/\frak p)$.
\end{theorem}
\begin{proof}
    We have $0 \subseteq \frak p \subseteq R$. As $R$ is catenary, a maximal chain in $R$ includes $\frak p$, so we are done by correspondence.
\end{proof}

This definition becomes more grounded when considering the fact that finite type algebras over a field are catenary. We don't include the proof.
We proceed now with a geometric characterization of a UFD.

\begin{theorem}
    Let $R$ be a noetherian domain. Then $R$ is a UFD if and only if all codimension 1 prime ideals are principal. To give some more exposition, a codimension 1 prime ideal in a domain is a minimal nonzero prime ideal. This corresponds to a maximal proper irreducible closed subset of $\spec R$. Note that $\spec R$ is irreducible in a domain. This condition then says that the geometric interpretation of a noetherian UFD is that any maximal proper irreducible closed subset of $\spec R$ is defined by one equation.
\end{theorem}
\begin{proof}
    Let $R$ be a noetherian UFD. Let $\frak p$ be a codimension 1 prime ideal. This says that $0 \subsetneq \frak p$ is a maximal chain. Let $x \neq 0$ in $\frak p$. As $R$ is a UFD, $x = f_1 \dots f_r$ irreducibles. By primality, some $\f_i \in \frak p$. Then we have $0 \subsetneq (f_i) \subseteq \frak p$. As $R$ is a UFD, $(f_i)$ is prime, so $(f_i) = \frak p$ as desired. Note that this did not use the noetherian condition.\\
    Conversely, suppose all codimension 1 prime ideals are principal. First, let $f \in R$ be a nonzero nonunit. Suppose that $f$ has no irreducible decomposition. In particular, $f$ is reducible, so $f = f_1 g_1$ for $f_1, g_1$ nonzero nonunits. Then at least one of these has no irreducible factorization. Without loss of generality, $f_1$ has no irreducible factorization. Then $f_1$ is a nonzero nonunit with no irreducible factorization. These are the same assumptions we had on $f$, so we repeat this process to yield $f_1 = f_2 g_2$ nonzero nonunits with $f_2$ having no irreducible factorization. This yields the chain $(f) \subseteq (f_1) \subseteq (f_2) \subseteq \dots$. These inclusions are strict as the $g_i$ are also assumed to be nonunits. This contradicts noetherianitude, so all nonzero nonunits have an irreducible factorization. Note that this argument holds in any noetherian ring.\\
    We now put the U in UFD. We first claim that for $f \in R$ irreducible that $(f)$ is prime. Indeed, $f$ is not a unit so $(f) \subsetneq R$. One can show that any nonempty set of prime ideals in a noetherian ring contains a minimal element. We show that any decreasing chain of primes stabilizes, which certainly implies the minimality condition. Indeed, let $\frak p_1 \supseteq \frak p_2 \supseteq \dots$ be a chain of primes. By noetherinitude, $\frak p_1 = (x_1, \dots, x_n)$. Then by Krull's principal ideal theorem (or rather its corollary), $\codim(\frak p_1) \leq n$. In particular, any chain of primes ending in $\frak p_1$ stabilizes, so this chain does. Now, we can let $\frak p \supseteq (f)$ be a minimal prime. By Krull's principal ideal theorem, it has codimension at most one. As $0 \subsetneq (f) \subseteq \frak p$, the codimension of $\frak p$ is 1. Thus, by assumption, it's principal so $\frak p = (g)$. Then $f = gh$ for some $h \in R$. As $f$ is irreducible and $g$ is not a unit, $h$ is a unit so $(f) = (g)$ is prime. Now let $f_1 \dots f_r = g_1 \dots g_s$ irreducibles. Then $g_1 \dots g_s \in (f_1)$, which is prime as above. Then some $g_i \in (f_1)$. As these are irreducible, $g_i \approx f_1$. The result follows by induction.
\end{proof}
\begin{remark}
    For a noetherian normal domain $R$, one can define an abelian group called the divisor class group CL$(R)$ which is generated by all codimension 1 prime ideals such that CL$(R)$ = 0 if and only if all codimension 1 primes are principal if and only if $R$ is a UFD. This measures the obstruction to being a UFD.
\end{remark}