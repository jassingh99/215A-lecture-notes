
\begin{definition}
    Let $A \subseteq B$ an extension of rings. We say $x \in B$ is integral over $A$ is it is the root of a monic polynomial in $A$. Of course, for fields algebraic is equivalent to integral.
\end{definition}

To gain motivation for this, consider the extension of rings $\mathbb C[x] \subseteq \mathbb C[x, y]$. Let $f \in \mathbb C[x, y]$. If $f$ is monic over $\mathbb C[x]$, then $f = y^n + a_{n-1}(x)y^{n-1} + \dots +a_0(x)$ for $a_i(x) \in \mathbb C[x]$. This tells us that after evaluating at any value of $x$, $f$ has exactly $n$ roots counting multiplicity. This is not the case whn $f = xy - 1$, for instance. Indeed, at $x = 0$ there are no roots and at $x = 1$ there is one root.

\begin{example}
    Let $K$ be a number field, i.e. a finite extension of $\mathbb Q$. We define $\mathcal O_K = \{x \in K : x \text{ is integral over } \mathbb Z\}$. This is called the ring of algebraic integers of $K$. We need to justify calling this a ring, which the following lemma will allow us to do.
\end{example}

\begin{lemma}
    Let $A \subseteq B$ an extension of rings. The following are equivalent.
    \begin{enumerate}
        \item $x \in B$ is integral over $A$.
        \item The sub $A$-algebra of $B$ generated by $x$ is finite over $A$ (finitely generated as an $A$-module). Call this algebra $C$.
        \item The algebra $C$ described above is contained in some finite $A$-algebra $D$.
        \item There is a faithful (scalar multiplication is trivial if and only if the scalar is $0$) $C$-module $M$ that is finitely generated as an $A$-module.
    \end{enumerate}
\end{lemma}
\begin{proof}
    $1) \implies 2)$ Let $x \in B$ integral over $A$. Then $x^n + a_{n-1} x^{n-1} + \dots + a_0 = 0$ for $a_i \in A$. Then $x^n \in A\{1, x, \dots, x^{n-1}\} = C$, so $C$ is a finite $A$-algebra.\\
    $2) \implies 3)$ Take $D = C$.\\
    $3) \implies 4)$ Let $M = D$. Then $M$ is a $C$-module that is a finite $A$-algebra. Thus, it is a finitely generated $A$-module. Furthermore, as $1 \in D$, the module is faithful.\\
    $4) \implies 1)$ Take $C$ and $M$ as given. Then $M = A\{m_1, \dots, m_n\}$. Then as $x \in C$ and $M$ is a $C$-module, each $xm_i \in M$. Then there exist $a_{ij}$ for each $i$ such that $xm_i = \sum_{i=1}^n a_{ij}m_j$. Let $Y = xI - (a_{ij})$ an $n \times n$ matrix with entries in $C$. Then let $m = (m_1, \dots, m_n)$. The definition of the $a_{ij}$ induce the equation $Ym = 0$. We have that $Y \text{adj}(Y) = \text{adj}(Y) Y = \det(Y) I$. Then $\det(Y)m = 0$. Thus, $\det(Y)$ kills the generators of $M$, so it kills all of $M$. Thus, $\det(Y) = 0$. $\det(Y)$ is a monic polynomial in $x$ with coefficients in $A$, so $x$ is indeed integral over $A$.
\end{proof}

\begin{definition}
    A ring homomorphism $A \longrightarrow B$ is called integral if $B$ is integral over its image. Similarly, it is called finite if $B$ is finite (i.e. a finitely generated module) over its image, and called finite type if $B$ is finite time (i.e. a finitely generated algebra) over the its image.
\end{definition}

\begin{corollary}
    A ring homomorphism $A \longrightarrow B$ is finite if and only if it is integral and of finite type.
\end{corollary}
\begin{proof}
    Finiteness certainly implies the other two properties. Conversely, suppose $A \longrightarrow B$ is finite type and integral. Assume without loss of generality that this map is the inclusion. Then $B = A[b_1,\ dots, b_n]$. By assumption, each $b_i$ is integral over $A$. Then by condition 2 in the lemma, $A[b_1]$ is finite over $A$. The result then follows by induction, as $b_2$ is certainly integral over $A[b_1]$.
\end{proof}

\begin{lemma}
    Let $A \subseteq B$ an extension of rings. Let $C$ be the integral closure of $A$ in $B$, i.e. all the elements of $B$ that are integral over $A$, is a subrng of $B$.
\end{lemma}
\begin{proof}
    Indeed, let $x, y \in C$. Then $x$ and $y$ both satisfy monic polynomials over $A$. Let $D$ be the $A$-algebra generated by $x^iy^j$ for $i \leq$ the degree of the monic polynomial $x$ satisfies and $j \leq$ the degree of the monic polynomial $y$ satisfies. Then $D$ is a finite $A$-algebra so all of its elements are integral over $A$ as the algebra they generate is contained in $D$. These elements include $x + y$, $x - y$, and $xy$. Thus, $C$ is indeed a ring.
\end{proof}

\begin{definition}
    A domain $R$ is called normal if it is integrally closed (i.e. equals its integral closure) in its quotient field.
\end{definition}

\begin{examples}
    \begin{enumerate}
        \item The ring of algebraic integers $\mathcal O_K$ is normal, as its quotient field is $K$ and it is already defined as the integral closure of $\mathbb Z$ in $K$.
        \item Any UFD is normal.
    \end{enumerate}
\end{examples}

\begin{lemma}
    Let $A \subseteq B$ be integral. Then 
    \begin{enumerate}
        \item For any ideal $I \subseteq B$, $A/(A \cap I) \longrightarrow B/I$ is integral.
        \item For $A \subseteq A$ a multiplicative subset, $A[S^{-1}] \subseteq B[S^{-1}]$ is integral.
    \end{enumerate}
\end{lemma}

\begin{lemma}
    Let $A \subseteq B$ an extension of rings. Let $C$ be the integral closure of $A$ in $B$. Then $C[S^{-1}]$ is the integral closure of $A[S^{-1}]$ in $B[S^{-1}]$.
\end{lemma}

\begin{corollary}
    Let $A$ be a domain. The following are equivalent.
    \begin{enumerate}
        \item $A$ is normal.
        \item All localizations of $A$ at prime ideals are normal.
        \item All localizations of $A$ at maximal ideals are normal.
    \end{enumerate}
    Which says that normality is a local property.
\end{corollary}
\begin{proof}
    Note that $\Frac(A[S^{-1}]) = \Frac(A)$.\\
    $1) \implies 2)$. Let $A$ be normal. Let $K$ be its quotient field. Let $\frak p \subseteq A$ a prime ideal. Then the integral closure of $A_{\frak p}$ in $\Frac(A_{\frak p}) = K$ is $A_{\frak p}$ by the lemma.\\
    $2) \implies 3)$. Clear.\\
    $3) \implies 1)$. Let $C$ be the integral closure of $A$ in $K$. We want to show that $A = C$. Equivalently, we want to show that the inclusion map $A \longrightarrow C$ is surjective. We know that a map being surjective is a local property. Furthermore, by assumption, each inclusion $A_{\frak m} \longrightarrow C_{\frak m}$ is surjective, as these domains are normal and by the lemma $C_{\frak m}$ is the integral closure of $A_{\frak m}$ in $K$. Hence, the inclusion $A \longrightarrow C$ is surjective as desired.
\end{proof}

\begin{example}
    Let $K$ be a umber field. As discussed above, $\mathcal O_K$ is normal, but is not in general a UFD. Indeed, one may show that the localization of $\mathcal O_K$ at any maximal ideal is a discrete valuation ring, hence a PID, hence a UFD, hence normal.
\end{example}